\documentclass[11pt,oneside]{article}
\usepackage{amsmath,amsfonts,amssymb}
%\usepackage{pst-plot, pstricks,pstricks-add}
\usepackage{fancyhdr}
\usepackage{framed}
\usepackage{multirow}
\usepackage{tikz}
\usetikzlibrary{arrows,positioning,shapes,fit,calc}
\usepackage{xcolor}
\pgfdeclarelayer{background}
\pgfsetlayers{background,main}
\usepackage{theorem}
%\usepackage{amsthm}
\usepackage{bbm}
\usepackage[latin1]{inputenc}
\usepackage{color}
\usepackage{epsfig}
%\usepackage{showkeys}  Kann man aus-/einblenden, um Labels im PDF anzuzeigen
%\setlength{\oddsidemargin}{0.5cm}
%\setlength{\evensidemargin}{0.5cm} %\setlength{\textwidth}{13cm}

%\newtheorem{thm}{Theorem}
%\newtheorem{prob}{Problem}
%\newcommand{\proof}{\noindent \textbf{Proof: }}
%\newcommand{\qed}{\hspace*{\fill} $\blacksquare$ }

\addtolength{\voffset}{-1in}
\addtolength{\oddsidemargin}{-1.9in}
\addtolength{\evensidemargin}{-.9in}
\addtolength{\textwidth}{1.5in}
\addtolength{\marginparwidth}{-2in}
\addtolength{\topmargin}{-2.3in}
\addtolength{\footskip}{1in}
\addtolength{\textheight}{4in}
\parindent 0pt
\addtolength{\headheight}{30pt}

\parindent 0pt
\raggedbottom

\usepackage{a4wide}
%\usepackage{indentfirst}
\usepackage{amsmath, systeme}
\usepackage{hanging}
\usepackage{graphicx}
\usepackage{float}
\graphicspath{ {./} }

\title{Paper Title}
\author{Alex Chkodrov, Saransh Kacharia}
\date{June, 7th 2019}

\fancyhf{}
\fancyhead[LE,RO]{Saransh Kacharia, Abishek Hariharan, Alex Chkodrov}
\fancyhead[RE,LO]{Paper Title}

\begin{document}
	\maketitle
	\pagestyle{fancy}
	\section{Abstract}
	\section{Problem Description}
	The Lotka-Voltera model, also known as the predator-prey equations, is a system of two differential equations that model the population change of two species in a predator and prey relationship. Volterra was an Italian mathematician that created the linearized solution to a two species system as an oscillating model representing the cyclic behavior of the species' populations. At the same time that Volterra created the model to represent predator prey populations, the American statistician Alfred Lotka proposed the same differential model to represent the  concentrations of the chemicals in an oscillating chemical reaction. The ubiquity of these interdependent relationships in the natural world has made the Lotka-Volterra system incredibly apt as a model for many scenarios.
	
	The original Lotka-Volterra equations:
	\begin{equation}
	\systeme{
		\frac{dx}{dt} = ax - bxy,
		\frac{dy}{dt} = -cy + dxy
	}
	\end{equation}
	\begin{itemize}
		\item $x(t)$ and $y(t)$ represent the population of the prey and predator populaitons respectively as functions of time.
		\item $a$ represents the growth rate constant of the prey population in the absence of predators.
		\item $b$ represents the prey's interactivity constant between the prey and the predator populations.
		\item $c$ represents the predator's natural decay rate in the absence of prey.
		\item $d$ represent's the predator's interactivity constant between the prey and the predator populations.
	\end{itemize}
	
	
	In this paper, A slightly more complex version of the Lotka-Volterra system involving three interacting species will be investigating. The added complexity makes this model slightly more applicable to the modeling of real populations, as the interconnected nature of ecosystems makes it very rare to encounter a two-species food chain. ???Add more
	
	
	\section{Simplifications and Mathematical Model}
	
	Population dynamics are incredibly complex, and in order to create a differential model for the interactions of a three-species food chain, many assumptions must be made regarding the environment and factors that influence population growth. The first assumption to make this model is that no other factor besides the population of other species in the food-chain can affect the populations' growth; seasonality, habitat loss, and human impact are all examples of realistic factors that can impact population growth that are not accounted for in this model.
	
	One key assumption in this model is that the rate of interaction between two species can be interpreted as the product of the two populations multiplied by some interactivity constant. It follows common sense when considering the interaction between two populations; when both populations are large, the product is large. When both populations are small, the product is small. When one population is large and one population is small, the product is somewhere in between. Thus, one can adequately model the interaction rate by the product of two populations.\\
	
	Another assumption is that the growth of the primary producer, for example plankton, is unbounded by the confines of the ocean or nutrients. Because of this assumption, this model cannot be reliably applied to environments which are near their carrying capacity. The growth of the primary producer depends on its current population multiplied by the growth rate constant. The decline of the population of the primary producer depends on the population of its predator multiplied by the primary producer's population and the interactivity constant. In this model, the only detractor of the primary producer's population is the secondary consumer, for example a filter-feeding fish, which preys on the primary producer; as mentioned above, it does not account for things such as the inavailability of nutrients and other constraints for the population growth of the primary producer.
	
	\begin{equation}
	\frac{dx}{dt} = ax - bxy
	\end{equation}
	\begin{itemize}
		\item $x(t)$ and $y(t)$ represent the populations of the primary producer and the secondary consumer respectively as functions of time.
		\item $a$ represents the growth rate constant of the primary producer in the absence of the secondary consumer.
		\item $b$ represents the primary producer's interactivity constant for the interaction of the primary producer and the secondary consumer.
	\end{itemize}
	
	The secondary consumer's population growth is reliant on the availability of food, represented by its own population multiplied by the population of its prey and their interactivity constant. The secondary consumer's population declines under the absence of food, requiring the addition of the decay rate constant multiplied by its own population. The secondary consumer's population also declines under predation, which depends on its own population multiplied by the predator's population and the interactivity constant. The magnitude of these changes depends on the size of the populations, which explains the use of products.
	
	\begin{equation}
	\frac{dy}{dt} = cxy - ey -dyz
	\end{equation}
	\begin{itemize}
		\item $x(t)$, $y(t)$ and $z(t)$ represent the populations of the primary producer, secondary consumer and the apex predator respectively as functions of time.
		\item $c$ represents the secondary consumer's interactivity constant for the interaction of the secondary consumer and the primary producer.
		\item $e$ represents the decay rate constant of the secondary producer in the absence of its prey, the primary producer.
		\item $d$ represents the secondary consumer's interactivity constant for the interaction of the secondary consumer and the apex predator.
	\end{itemize}
	
	The secondary consumer's population growth is reliant on the availability of food, represented by its own population multiplied by the population of its prey and their interactivity constant. The apex predator's population declines in the absence of food, requiring the addition of the decay rate constant multiplied by its population. Again, the assumption is that things that affect real life apex predators, for example mercury poisoning in tuna, are not factored into this model.
	
	\begin{equation}
	\frac{dz}{dt} = fyz - gz
	\end{equation}
	\begin{itemize}
		\item $y(t)$ and $z(t)$ represent the populations of the secondary consumer and the apex predator respectively as functions of time.
		\item $f$ represents the decay rate constant of the apex predator in the absence of its prey, the secondary consumer.
		\item $g$ represents the apex predator's interactivity constant for the interaction of the apex predator and the secondary consumer.
	\end{itemize}
	
	This simplified model is how the lotka-volterra system would be represented for a three-species food chain with no outside interference or carrying capacity. The complete system of differential equations for this model:
	
	\begin{equation}
	\systeme{
		\frac{dx}{dt} = ax - bxy,
		\frac{dy}{dt} = cxy - ey -dyz,
		\frac{dz}{dt} = fyz - gz
	}
	\end{equation}
	\begin{itemize}
		\item $x(t)$, $y(t)$ and $z(t)$ represent the population of the primary producer, secondary consumer and apex predator respectively as functions of time.
		\item $a$ represents the growth rate constant of the primary producer in the absence of the secondary consumer.
		\item $b$ represents the primary producer's interactivity constant for the interaction of the primary producer and the secondary consumer.
		\item $c$ represents the secondary consumer's interactivity constant for the interaction of the secondary consumer and the primary producer.
		\item $e$ represents the decay rate constant of the secondary producer in the absence of its prey, the primary producer.
		\item $d$ represents the secondary consumer's interactivity constant for the interaction of the secondary consumer and the apex predator.
		\item $f$ represents the decay rate constant of the apex predator in the absence of its prey, the secondary consumer.
		\item $g$ represents the apex predator's interactivity constant for the interaction of the apex predator and the secondary consumer.
	\end{itemize}
	
	
	\section{Solution to the Mathematical Problem}
	To solve this mathematical model we first find the critical points at which the simultaneous differential equations are all equal to zero, and then we use linear analysis to determine the stability of each critical point. 
	\begin{equation}
	\begin{cases}
	\frac{dx}{dt} = ax - bxy = 0\\
	\frac{dy}{dt} = cxy - ey -dyz = 0\\
	\frac{dz}{dt} = fyz - gz = 0
	\end{cases}
	\end{equation}
	\begin{equation}
	\begin{cases}
	\frac{dx}{dt} = x(a-by) = 0\\
	\frac{dy}{dt} = y(cx - e -dz) = 0\\
	\frac{dz}{dt} = z(fy - g) = 0
	\end{cases}
	\end{equation}
	
	We find the null clines,
	\begin{align}
	\frac{dx}{dt} &= 0 \rightarrow x = 0 \text{ or } y = \frac{a}{b}\\
	\frac{dy}{dt} &= 0 \rightarrow y=0 \text{ or } x = \frac{dz + e}{c} \text{ and } z = \frac{cx - e}{d}\\
	\frac{dz}{dt} &= 0 \rightarrow z = 0 \text{ or } y = \frac{g}{f}
	\end{align}
	
	If we set $x$ equal to zero, we get 
	\begin{equation}
	\begin{cases}
	\frac{dx}{dt} = 0\\
	\frac{dy}{dt} = -dyz - ey\\
	\frac{dz}{dt} = fyz - gz
	\end{cases}
	\end{equation}
	\begin{figure}[H]
		\center
		\includegraphics[scale=0.5]{YZ}
		\caption{Pls Caption Me}
	\end{figure}
	%YX Plot
	%Description and reasoning
	
	If we set $y$ equal to zero, we get
	\begin{equation}
	\begin{cases}
	\frac{dx}{dt} = ax\\
	\frac{dy}{dt} = 0\\
	\frac{dz}{dt} = gz
	\end{cases}
	\rightarrow
	\end{equation}
	\begin{figure}[H]
		\center
		\includegraphics[scale=0.5]{XZ}
		\caption{Pls Caption Me}
	\end{figure}
	%x(t) = ..., z(t) = ....
	%ZX plot
	%Description and reasoning
	
	If we set $z$ equal to zero, we get
	\begin{equation}
	\begin{cases}
	\frac{dx}{dt} = ax - bxy\\
	\frac{dy}{dt} = cxy - ey\\
	\frac{dz}{dt} = 0
	\end{cases}
	\end{equation}
	
	For the critical point ($\frac{e}{c}$, $\frac{a}{b}$, $0$), we can determine the behavior of the system by linearizing it at the critical point and examining the eigenvalues of the Jacobian of the linearized system.
	
	\begin{equation}
	J(x,y) = 
	\begin{bmatrix}
	\frac{\partial}{\partial x}(ax - bxy) && \frac{\partial}{\partial y}(ax - bxy) \\
	\frac{\partial}{\partial x}(cxy - ey) && \frac{\partial}{\partial y}(cxy - ey)
	\end{bmatrix}
	\end{equation}
	%Explain why J(X,Y) instead of J(X,Y,Z) because dz/dt is zero
	
	\begin{equation}
	J(x,y) = 
	\begin{bmatrix}
	a - by && -bx \\
	cy && cx - e
	\end{bmatrix}
	\end{equation}
	
	%Plugging in the critical point for the jacobian
	\begin{equation}
	J(\frac{e}{c},\frac{a}{b}) = 
	\begin{bmatrix}
	0 && \frac{be}{c} \\
	\frac{ca}{b} && 0
	\end{bmatrix}
	\end{equation}
	
	The eigenvalues for the Jacobian matrix at the point ($\frac{e}{c}$, $\frac{a}{b}$, $0$) are,
	\[\lambda = \pm i\sqrt{ae}\]
	
	\begin{figure}[H]
		\center
		\includegraphics[scale=0.5]{XY}
		\caption{Pls Caption Me}
	\end{figure}
	%XY plot
	%Looks like the classic Lotka Volterra model
	%Description and reasoning
	
	Cumulative plot
	\begin{figure}[H]
		\center
		\includegraphics[scale=0.80]{XYZ}
		\caption{Pls Caption Me}
	\end{figure}
	
	
	\section{Results and Discussion}
	
	\section{Improvement}
	
	\section{Conclusions}
	
	\section{References}
	\begin{hangparas}{.25in}{1}
		Chauvet, Erica, et al. "A Lotka-Volterra Three-Species Food Chain." \textit{Mathematics Magazine}, Oct. 2002, pp. 243-255.
		
		Lotka, Alfred J. \textit{Elements of Mathematical Biology}. Dover Publications, 1956.
		
		Boyce, William E., and Richard C. DiPrima. \textit{Elementary Differential Equations and Boundary Value Problems}. 10th ed., John Wiley \& Sons, Inc., 2017.
		
		
	\end{hangparas}
	
\end{document}