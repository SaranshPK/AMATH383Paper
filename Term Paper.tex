\documentclass[11pt,oneside]{article}
\usepackage{amsmath,amsfonts,amssymb}
%\usepackage{pst-plot, pstricks,pstricks-add}
\usepackage{fancyhdr}
\usepackage{framed}
\usepackage{multirow}
\usepackage{tikz}
\usetikzlibrary{arrows,positioning,shapes,fit,calc}
\usepackage{xcolor}
\pgfdeclarelayer{background}
\pgfsetlayers{background,main}
\usepackage{theorem}
%\usepackage{amsthm}
\usepackage{bbm}
\usepackage[latin1]{inputenc}
\usepackage{color}
\usepackage{epsfig}
%\usepackage{showkeys}  Kann man aus-/einblenden, um Labels im PDF anzuzeigen
%\setlength{\oddsidemargin}{0.5cm}
%\setlength{\evensidemargin}{0.5cm} %\setlength{\textwidth}{13cm}

%\newtheorem{thm}{Theorem}
%\newtheorem{prob}{Problem}
%\newcommand{\proof}{\noindent \textbf{Proof: }}
%\newcommand{\qed}{\hspace*{\fill} $\blacksquare$ }

\addtolength{\voffset}{-1in}
\addtolength{\oddsidemargin}{-1.9in}
\addtolength{\evensidemargin}{-.9in}
\addtolength{\textwidth}{1.5in}
\addtolength{\marginparwidth}{-2in}
\addtolength{\topmargin}{-2.3in}
\addtolength{\footskip}{1in}
\addtolength{\textheight}{4in}
\parindent 0pt
\addtolength{\headheight}{30pt}

\parindent 0pt
\raggedbottom

\usepackage{a4wide}

\title{Paper Title}
\author{Saransh Kacharia, Abishek Hariharan, Alex Chkodrov}
\date{June, 7th 2019}

\fancyhf{}
\fancyhead[LE,RO]{Saransh Kacharia, Abishek Hariharan, Alex Chkodrov}
\fancyhead[RE,LO]{Paper Title}

\begin{document}
\maketitle
\pagestyle{fancy}
\section{Abstract}
\section{Problem Description}
The Lotka-Voltera equations, also known as the predator-prey equations, are equations that model two species in a predator and prey relationship in an ecosystem. The linearized solution to this two species system is an oscillating model that 
In 1925, the American statistician Alfred Lotka proposed a differential model to represent the concentrations of chemicals in a theoretical oscillating chemical reaction ? The same differential equations were proposed independently by the Italian mathematician Vito Volterra as a model for oscillating populations representing the populations of a predator and its prey. This model can be used to identify what is the main cause and reaction towards the overall
\section{The Model}
Population dynamics are incredibly complex, and in order to create a differential model for the interactions of a three-species food chain, many assumptions must be made regarding the environment and factors that influence population growth. 
In this admittedly unrealistic population model, the growth of the primary producer, for example plankton, is unbounded by the confines of the ocean or nutrients ? only the current population and growth rate, and the population of its predator. This applies practically to examples that have not reached a carrying capacity and/or developing populations. The only detractor of the population of the primary producer is the secondary consumer, for example a filter-feeding fish, which preys on the primary producer.\\ 
*Insert first equation*
The population of the secondary consumer depends on its current population, growth rate, the population of its prey (the primary producer), and the population of the apex predator. The assumption is that no other factor besides the population of the rest of the species in the food-chain affects the population of the secondary consumer; seasonality, habitat loss, and human impact are all examples of realistic factors that can impact population growth that are not accounted for in this model. \\
*Insert second equation*
The population of the apex predator depends on its current population, growth rate, and the population of its prey. Again, the assumption is that things that affect real life apex predators like mercury poisoning in tuna are not factored into this model; this assumes no outside interference in the population of either the apex predator itself or its prey.\\
*Insert third equation*
This simplified model is how the lotka-volterra system would be represented for a three-species food chain, assuming no outside interference or carrying capacity.


\end{document}