\documentclass[11pt,oneside]{article}
\usepackage{amsmath,amsfonts,amssymb}
%\usepackage{pst-plot, pstricks,pstricks-add}
\usepackage{fancyhdr}
\usepackage{framed}
\usepackage{multirow}
\usepackage{tikz}
\usetikzlibrary{arrows,positioning,shapes,fit,calc}
\usepackage{xcolor}
\pgfdeclarelayer{background}
\pgfsetlayers{background,main}
\usepackage{theorem}
%\usepackage{amsthm}
\usepackage{bbm}
\usepackage[latin1]{inputenc}
\usepackage{color}
\usepackage{epsfig}
%\usepackage{showkeys}  Kann man aus-/einblenden, um Labels im PDF anzuzeigen
%\setlength{\oddsidemargin}{0.5cm}
%\setlength{\evensidemargin}{0.5cm} %\setlength{\textwidth}{13cm}

\newtheorem{thm}{Theorem}
\newtheorem{prob}{Problem}
\newcommand{\proof}{\noindent \textbf{Proof: }}
\newcommand{\qed}{\hspace*{\fill} $\blacksquare$ }



%\pagestyle{fancy}
%\fancyhf{}
%\fancyhead[LE,RO]{\thepage}
%\fancyhead[RE,LO]{FINAL MATH328}
	\addtolength{\voffset}{-1in}
	\addtolength{\oddsidemargin}{-1.9in}
	\addtolength{\evensidemargin}{-.9in}
	\addtolength{\textwidth}{1.5in}
	\addtolength{\marginparwidth}{-2in}
	\addtolength{\topmargin}{-2.3in}
		\addtolength{\footskip}{1in}
	\addtolength{\textheight}{4in}
\parindent 0pt
\addtolength{\headheight}{30pt}

\pagestyle{fancy}
\fancyhf{}
\fancyhead[LE,RO]{MATH424 HW8}
\fancyhead[RE,LO]{Abishek Hariharan\\ University of Washington}

\parindent 0pt
\raggedbottom
%\raggedbottom
%++++++++++++++++++++++++++++++++++++++++++++++++++++++++++++++++++++++++++++++
%++++++++++++++++++++++++++++++++++++++++++++++++++++++++++++++++++++++++++++++

\renewcommand{\labelenumi}{(\alph{enumi})}                  % enumerate: (a) (i)
%\renewcommand{\labelenumii}{(\roman{enumii})}             %                          (ii) ...
\usepackage{a4wide}
\begin{document}
%+++++++++++++++++++++++++++++++++++++++++++++++++++++++++++++++++++++++++++++++
%+++++++++++++++++++++++++++++++++++++++++++++++++++++++++++++++++++++++++++++++
%+++++++++++++++++++++++++++++++++++++++++++++++++++++++++++++++++++++++++++++++


%+++++++++++++++++++++++++++++++++++++++++++++++++++++++++++++++++++++++++++++++
\section*{Homework 8} 
\paragraph{Active Learning}

\begin{prob}
	Nothing this week.
\end{prob}

///////////////////////////////////////////////////////////////////////////////

\paragraph{General Homework}

\begin{prob}
Find the radius of convergence for the following series

	 $\sum_{k=0}^{\infty}2^kx^{2k}$, $\sum_{k=1}^{\infty}\frac{(-1)^k}{2^{k+1}\sqrt k}x^k$,  
	 $\sum_{k=0}^{\infty}(k^2+2^k)x^k$, $\sum_{k=1}^{\infty}{2k\choose k}(x+1)^{2k}$,
	 $\sum_{k=1}^{\infty}\frac{1}{(4+(-1)^k)^{3k}}x^{5k+1}$, $\sum_{k=0}^{\infty}a^{k!}x^k, \:\:a>0$, and
	 $\sum_{k=0}^{\infty}\left(\frac{k!}{3\cdot5\cdot7\cdot\cdot\cdot(2k+1)}\right)^2x^k$.

\end{prob}

\begin{prob}
	\begin{enumerate}
		\item Show that $\sum_{k=0}^{\infty}\frac{(-1)^k}{(2k)!}x^{2k}$ is convergent for all $x\in\mathbb R$.
		\item Show that for the function $f(x)=\sum_{k=0}^{\infty}\frac{(-1)^k}{(2k)!}x^{2k} $, $f''(x)+f(x)=0$ for
			 all $x\in\mathbb R$ 
		\item Which function has $\sum_{k=0}^{\infty}\frac{(-1)^k}{(2k)!}x^{2k}$ as power series expansion? (Use 307)
	\end{enumerate}
\end{prob}

\begin{prob}
	For each $R\in[0,\infty)\cup\{\infty\}$, find a power series with radius of convergence $R$.
\end{prob}

\begin{prob}
	\begin{enumerate}
		\item Find the power series expansion of $\arctan(x)$ for $|x|<1$, i.e. find how to write $\arctan(x)$ in the 
			form $\sum_{k=0}^{\infty}a_kx^k$. Use a similar approach as in Example 4.3.4.
		\item Write $\ln(1-2x)$ as a power series.
	\end{enumerate}
\end{prob}

\begin{prob}%6.4.21
	Let $f(x)=\sum_{k=0}^{\infty}a_kx^k$ be a power series such that $a_k\in\mathbb Z$ for all $k\in\mathbb N$. Assume that its 
	radius of convergence is $R>1$. Show that $f$ is a polynomial, i.e. there is $N\in\mathbb N$ such that $a_n=0$ for all $n\ge N$. 
\end{prob}

\begin{prob}
	Let $\sum_{k=0}^{\infty}a_kx^k$ be a power series with radius of convergence $R\in(0,\infty)$. Show that 
	$B(X)=\sum_{k=0}^{\infty}a_kx^{k^2}$ has radius of convergence 1.
	{\it (Hint First prove that $R_B\le 1$. To show the other direction, make use of the fact that $\limsup c_n=c$ implies that
	there is a subsequence $c_{k_j}$ such that $c_{j_k}\rightarrow c$)}.
\end{prob}

\end{document}